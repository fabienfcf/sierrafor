% latex table generated in R 4.5.1 by xtable 1.8-4 package
% Thu Dec 11 12:03:42 2025
\begin{table}[ht]
\centering
\begin{tabular}{llrrrrr}
  \toprule
nombre_cientifico & genero_grupo & densidad_ha & proporcion_pct & d_medio_cm & dmc_cm & ab_m2ha \\ 
  \midrule
Quercus laeta & Quercus & 124.7 & 35.76 & 21.96 & 23.38 & 5.35 \\ 
  Quercus rysophylla & Quercus & 120.9 & 34.67 & 23.10 & 24.25 & 5.58 \\ 
  Pinus pseudostrobus & Pinus & 61.9 & 17.75 & 25.69 & 27.90 & 3.78 \\ 
  Pinus teocote & Pinus & 13.6 & 3.91 & 21.13 & 22.92 & 0.56 \\ 
  Juniperus flaccida & Juniperus & 8.6 & 2.47 & 18.12 & 19.28 & 0.25 \\ 
  Arbutus xalapensis & Arbutus & 7.1 & 2.03 & 19.95 & 21.14 & 0.25 \\ 
  Crataegus mexicana & Crataegus & 4.0 & 1.14 & 14.70 & 15.77 & 0.08 \\ 
  Quercus affinis & Quercus & 3.3 & 0.94 & 14.58 & 15.51 & 0.06 \\ 
  Juglans sp & Juglans & 3.1 & 0.89 & 18.39 & 19.61 & 0.09 \\ 
  Quercus laceyi & Quercus & 1.2 & 0.35 & 17.14 & 17.54 & 0.03 \\ 
   \bottomrule
\end{tabular}
\caption{Principales especies (top 10, incluye AB/ha y DMC)} 
\label{tab:top_especies}
\end{table}
