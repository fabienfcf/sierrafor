% latex table generated in R 4.5.1 by xtable 1.8-4 package
% Fri Nov  7 13:49:18 2025
\begin{table}[ht]
\centering
\caption{Ecuaciones alométricas utilizadas (Sección 7.2)} 
\label{tab:ecuaciones_alometricas}
\begin{tabular}{llrrrl}
  \toprule
Especie & Forma & a & b & c & Fuente \\ 
  \midrule
Quercus affinis & V = a × D^b × H^c & 0.0001 & 1.8560 & 0.9790 & SIPLAFOR \\ 
  Quercus laceyi & exp & -9.4869 & 1.8241 & 0.9689 & 1er INFYS \\ 
  Quercus rysophylla & exp & -9.4869 & 1.8241 & 0.9689 & 1er INFYS \\ 
  Quercus laeta & exp & -9.4869 & 1.8241 & 0.9689 & 1er INFYS \\ 
  Pinus cembroides & exp & -9.8208 & 1.8918 & 1.0805 & 1er INFYS \\ 
  Pinus pseudostrobus & V = a × D^b × H^c & 0.0000 & 1.9369 & 1.0317 & SIPLAFOR \\ 
  Pinus teocote & exp & -8.7264 & 1.4303 & 1.1954 & 1er INFYS \\ 
  Juniperus flaccida & V = a × D^b × H^c & 0.0001 & 1.7719 & 0.7349 & SIPLAFOR \\ 
  Arbutus xalapensis & V = a × D^b × H^c & 0.0001 & 1.8215 & 0.9612 & SIPLAFOR \\ 
  Fraxinus sp & exp & -9.8043 & 1.9103 & 1.0326 & 1er INFYS \\ 
  Carya sp & exp & -9.9822 & 1.9424 & 1.0223 & 1er INFYS \\ 
  Crataegus mexicana & V = a × D^b × H^c & 0.0001 & 1.9823 & 0.8422 & SIPLAFOR \\ 
  Juglans sp & exp & -9.8294 & 1.9060 & 1.0405 & 1er INFYS \\ 
   \bottomrule
\end{tabular}
\end{table}
